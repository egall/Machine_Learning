\documentclass[12pt]{article} % <--- Please use 12pt font
\usepackage{mathptmx}
\usepackage[T1]{fontenc}


% Please put your title here (not more than two lines)
\title{Blog Likeliness Prediction Based on User's ``Likes"}

% List the name of the speaker/presenter and any coauthors here 
% together with their affiliations
\author{\textbf{Rakshit Agrawal, Erik Steggall}\\ esteggall@soe.ucsc.edu, ragrawa1@ucsc.edu}


\date{} % <--- Please leave date empty

\begin{document}
\maketitle
\thispagestyle{empty}


\noindent \textbf{Abstract} \\
\noindent Please put your abstract here (not more than half a page)

\vspace{18pt}
\noindent \textbf{Keywords} - Mean Average Precision (MAP)


\noindent \textbf{Introduction} \\
\noindent 

% What is the general problem that we are addressing
Recent trends have lead to a large increase in the online presence of the average Internet user across the globe. Companies such as Google, Facebook, and Netflix store huge corpuses of data with detailed information about each user on their system, however without automated analysis these vast data corpuses are largely useless. There is a current effort to utilize the information stored in these corpuses in order to build a customized Internet tailored to each users individual preferences. At the core of this effort is machine learning, and in particular, the use of \textit{recommender systems}.\\
% What is a recommender system and how does it help solve his problem?
Recommender systems are intended to aid a user by recommending items that they may be interested in. Theoretically, recommender systems will aid a user in exploring the Internet which is becoming increasingly complex, and thus confusing for most users. In order to provide these suggestions, recommender systems use information about the users current preferences and information who share similar preferences. Recommender systems can be broken down into two main schools; content based recommender systems and collaborative recommender systems.\\
Content based recommender systems are designed around using the users past preferences in order to generate a list of items that may be of interest to them in the future. These systems examine similarities among items on the system and will recommend an item to a user if it shares a number of similarities with items that they have indicated a preference for in the past. Content based recommender systems are excellent at recommending items that a particular user may be interested based on their past, but are limited to that users past preference, thus, they are not good at predicting outside of a users current preference list which makes them prone to suggesting ``stale" items.\\
To contrast, collaborative recommender systems are designed around examining similarities among users. A collaborative recommender system will examine the relationships between user preference lists and suggest items to a user based on items that users with similar interests preferred. This allows the recommender system to suggest items that are outside of the users current preference list, thus providing spontaneity in the items that are suggested.\\
% Include hybrid system?
% What do we have in our specific case?
For this project we are implementing a hybrid of these two techniques in order to analyze a dataset of users and blogs to predict which blogs they are most likely to be interested in. \\
% What is our specific problem we are trying to solve 

\noindent \textbf{Methodology/Plans} \\
\noindent 
\textit{Dataset}\\
Our dataset is supplied through a Kaggle competition and is split into sets of users and sets of blogs that each contain dictionaries that link users and blogs if the blog was ``liked" by a user. The data was harvested from ~90,000 Wordpress blogs over the period of 6 weeks, we plan on splitting off the last week in order to produce a test set for our data as the existing test set on Kaggle does not have any ground truth for us to understand if we predicted accurately or not.\\

\textit{Preprocess}\\

\textit{Experiments and Evaluation}\\
In order to evaluate the performance of our algorithms we plan to use the measure of mean average precision or (MAP) to compare against our selected items.\\%Need to define this above.
We plan to create separate algorithms for collaborative and content based systems, and then merge the results to form a hybrid system. We will evaluate all three individually, but ultimately we are trying to improve the MAP scores on the hybrid algorithm which we expect to get the best results.\\ 

\textit{Parameter tuning}\\
%We plan to use the normal equation and Pearson Correlation Coefficient?

\textit{Learning methods and tools}\\
We are writing our algorithms in Python using NumPy. Since we are using multiple algorithms to design each recommender system we plan on writing them out in Python, then testing individual aspects of our code against libraries written in R. We are using an SQL database to store our dataset.\\
We plan to follow a similar path as the Bellkor team did for the Netflix 2007 challenge. In their writeup of the competition  \cite{bell07} they use KNN, singular value decomposition with ridge regression, and restricted Boltzmann machines. We plan to use KNN to cluster user-user and blog-blog groups that have similar attributes. \\

\noindent \textbf{Progress/Problems} \\
\noindent 
\textit{Progress}\\
So far we have done most of the background needed in determining the algorithms that we will be using for this project. We have also written high level pseudocode for our general problem.\\
We have also done a good amount of investigation of our dataset, including creating a SQL database to hold user and blog information. We have also written the code to parse the dataset and populate our database.\\
\textit{Problems}\\
We have a very large dataset and may have problems with the sparsity of likes in our dataset. This may cause problems with our matrix factorization, but from our reading we have found that the Bellkor team had similar problems and were successful in their works \cite{bell08}.\\
A related worry is about generating null suggestion lists, as they will have perfect accuracy, but do not fulfill the intent of our project.\\
We also need a more concrete plan as to how we will combine our algorithms to create a full recommender system. This may be difficult as neither of us was familiar with recommender systems before starting this project and we are still unclear on some of the components such as SVD factorization.\\

\begin{thebibliography}{1}
\bibitem{bell07}
  Robert M. Bell, Yehuda Koren, Chris Volinsky
  \emph{Modeling Relationships at Multiple Scales to Improve Accuracy of Large Recommender Systems}
  Florham Park, NJ
  2007
  
\bibitem{bell08}
  Robert M. Bell, Yehuda Koren, Chris Volinsky
  \emph{The BellKor 2008 Solution to the Netflix Prize}
  Florham Park, NJ
  2008
  
\bibitem{bell09}
  Robert M. Bell, Yehuda Koren, Chris Volinsky
  \emph{Matrix Factorization Techniques for Recommender Systems}
  Florham Park, NJ
  2009
  
\bibitem{koren08}
  Robert M. Bell, Yehuda Koren, Chris Volinsky
  \emph{The BellKor solution to the Netflix Prize}
  Florham Park, NJ
  2009

\bibitem{salakhutdinov07}
  Ruslan Salakhutdinov, Andriy Mnih, Geoffrey Hinton
  \emph{Restricted Boltzmann Machines for Collaborative Filtering}
  Toronto, Ontario
  2007
  
\bibitem{li03}
  Qing Li, Byeong Man Kim
  \emph{Clustering Approach for Hybrid Recommender System}
  Kumi, South Korea
  2003
\end{thebibliography}


\end{document}