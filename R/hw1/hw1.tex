\documentclass{article}

\usepackage[latin1]{inputenc}
\usepackage{times,fullpage,amsmath}
\usepackage{enumerate,graphicx,hyperref,verbatim, amsmath, pdfpages}

\title{TIM 209 Homework \#1}
\author{Erik Steggall\\ esteggall@soe.ucsc.edu}

\date{Fall 2014}
\setlength\parindent{0pt}
\begin{document} \maketitle \pagestyle{empty}
\section*{Problem 1}
\subsection*{a}

\begin{tabular}{ l c r }
x1 -- 3\\
x2 -- 2\\
x3 -- 3.162278\\
x4 -- 2.236068\\
x5 -- 1.414214\\
x6 -- 1.732051\\
\end{tabular}
\subsection*{b}
When $K=1$ our prediction is that it is the point will be Green. It will choose green because the observation closest to the origin x5, whose label is Green.\\
\subsection*{c}
When $K=3$ our prediction is the point will be Red. This is because, of the first three points; x5, x6, and x2, Red is the majority.\\
\subsection*{d}
A smaller boundary is best. Smaller values of K for a non-linear will be more accurate than a large value because it will fit curves better than a large K would.

\section*{Problem 2}

\includepdf[pages={1}]{hw1_prob2.pdf}

\section*{Problem 3}
\subsection*{ii}
\begin{verbatim}
> summary(college)
 Private        Apps           Accept          Enroll       Top10perc    
 No :212   Min.   :   81   Min.   :   72   Min.   :  35   Min.   : 1.00  
 Yes:565   1st Qu.:  776   1st Qu.:  604   1st Qu.: 242   1st Qu.:15.00  
           Median : 1558   Median : 1110   Median : 434   Median :23.00  
           Mean   : 3002   Mean   : 2019   Mean   : 780   Mean   :27.56  
           3rd Qu.: 3624   3rd Qu.: 2424   3rd Qu.: 902   3rd Qu.:35.00  
           Max.   :48094   Max.   :26330   Max.   :6392   Max.   :96.00  
   Top25perc      F.Undergrad     P.Undergrad         Outstate    
 Min.   :  9.0   Min.   :  139   Min.   :    1.0   Min.   : 2340  
 1st Qu.: 41.0   1st Qu.:  992   1st Qu.:   95.0   1st Qu.: 7320  
 Median : 54.0   Median : 1707   Median :  353.0   Median : 9990  
 Mean   : 55.8   Mean   : 3700   Mean   :  855.3   Mean   :10441  
 3rd Qu.: 69.0   3rd Qu.: 4005   3rd Qu.:  967.0   3rd Qu.:12925  
 Max.   :100.0   Max.   :31643   Max.   :21836.0   Max.   :21700  
   Room.Board       Books           Personal         PhD        
 Min.   :1780   Min.   :  96.0   Min.   : 250   Min.   :  8.00  
 1st Qu.:3597   1st Qu.: 470.0   1st Qu.: 850   1st Qu.: 62.00  
 Median :4200   Median : 500.0   Median :1200   Median : 75.00  
 Mean   :4358   Mean   : 549.4   Mean   :1341   Mean   : 72.66  
 3rd Qu.:5050   3rd Qu.: 600.0   3rd Qu.:1700   3rd Qu.: 85.00  
 Max.   :8124   Max.   :2340.0   Max.   :6800   Max.   :103.00  
    Terminal       S.F.Ratio      perc.alumni        Expend     
 Min.   : 24.0   Min.   : 2.50   Min.   : 0.00   Min.   : 3186  
 1st Qu.: 71.0   1st Qu.:11.50   1st Qu.:13.00   1st Qu.: 6751  
 Median : 82.0   Median :13.60   Median :21.00   Median : 8377  
 Mean   : 79.7   Mean   :14.09   Mean   :22.74   Mean   : 9660  
 3rd Qu.: 92.0   3rd Qu.:16.50   3rd Qu.:31.00   3rd Qu.:10830  
 Max.   :100.0   Max.   :39.80   Max.   :64.00   Max.   :56233  
   Grad.Rate     
 Min.   : 10.00  
 1st Qu.: 53.00  
 Median : 65.00  
 Mean   : 65.46  
 3rd Qu.: 78.00  
 Max.   :118.00
\end{verbatim}
\subsection*{iii}
\includepdf[pages={1}]{hw1_prob3c.pdf}
\includepdf[pages={1}]{hw1_prob3c2.pdf}

\subsection*{iv}
\includepdf[pages={1}]{hw1_prob3d.pdf}
\subsection*{vi\\summary of my findings}
\includepdf[pages={1}]{hw1_prob3f.pdf}

The grad rate and top 10 percent appear to be very loosely related. The graph is very sparse\\

Comparing acceptance to the top 10 shows that they are mostly similar, however there are a few outliers that are way different.\\

Acceptance vs expenditure have a strong connection, but similar to the acceptance vs top 10, it has a outliers that are significantly different.\\


\end{document}