\documentclass{article}

\usepackage[latin1]{inputenc}
\usepackage{times,fullpage,amsmath}
\usepackage{enumerate,graphicx,hyperref,verbatim, amsmath, mathtools, pdfpages}
\DeclareMathSizes{10}{10}{10}{10}
\title{TIM209 Homework \#5}
\author{Erik Steggall \\ esteggall@soe.ucsc.edu}

\date{Fall 2014}
\setlength\parindent{0pt}
\begin{document} \maketitle \pagestyle{empty}
\section*{Problem 1}
\subsection*{c}
\begin{verbatim}
Bic = [1] 3
cp = [1] 4
rsq = [1] 4
\end{verbatim}
\includepdf[pages={1}]{bic.pdf}
\includepdf[pages={1}]{cp.pdf}
\includepdf[pages={1}]{adjr2.pdf}

\section*{Problem 2}
\subsection*{c}
\includepdf[pages={1}]{trainingmse.pdf}
\subsection*{d}
\includepdf[pages={1}]{testmse.pdf}
\subsection*{e}
Model 15 has the lowest test MSE.\\
\subsection*{f}
\begin{verbatim}
 [1]  0.08416949  0.00000000 -0.74255920  0.00000000  1.14136590  0.83065524
 [7] -1.73567582  1.01058528  0.00000000  0.28645981 -0.51944548  0.00000000
[13] -0.29197205  0.16706164  0.00000000  1.21964261 -1.89792374  2.07617863
[19] -1.08201210  0.35911887
(Intercept)         x.1         x.3         x.5         x.6         x.7 
 -0.2202223   0.1101895  -0.6750294   0.9762085   0.8042316  -1.8396843 
        x.8         x.9        x.10        x.11        x.12        x.16 
  1.1182198   0.1247155   0.4014708  -0.3397521  -0.1960752   1.4241490 
       x.17        x.18        x.19        x.20 
 -1.7915828   1.8074764  -1.0706236   0.3124571 
 \end{verbatim}
The zeros are all pulled up and their values seem to be distributed equally over other coefficients.\\

\section*{Problem 3}

\subsection*{a}
\includepdf[pages={1}]{prob3a.pdf}


\subsection*{b}
The RSS decreases as the degree of the polynomial increases. This is expected as higher degree polynomials are more flexible.\\
\begin{verbatim}
All of the RSS values:  [1] 2.768563 2.035262 1.934107 1.932981 1.915290 1.878257 1.849484 1.835630 1.833331 1.832171
\end{verbatim}

\subsection*{c}

I would pick either a third degree polynomial. The graph starts to increase after the third degree, the fourth degree polynomial would have almost the same error, but it would also have a higher variance which would make it less desirable than the third degree polynomial.\\
\includepdf[pages={1}]{Problem3d.pdf}


\subsection*{d}
I picked the splines that I figured would be around the middle of the dataset.\\
\begin{verbatim}
Call:
lm(formula = nox ~ bs(dis, df = 4, knots = c(3, 6, 9)), data = Boston)

Residuals:
      Min        1Q    Median        3Q       Max 
-0.132134 -0.039466 -0.009042  0.025344  0.187258 

Coefficients:
                                      Estimate Std. Error t value Pr(>|t|)    
(Intercept)                           0.709144   0.016099  44.049  < 2e-16 ***
bs(dis, df = 4, knots = c(3, 6, 9))1  0.006631   0.025467   0.260    0.795    
bs(dis, df = 4, knots = c(3, 6, 9))2 -0.258296   0.017759 -14.544  < 2e-16 ***
bs(dis, df = 4, knots = c(3, 6, 9))3 -0.233326   0.027248  -8.563  < 2e-16 ***
bs(dis, df = 4, knots = c(3, 6, 9))4 -0.336530   0.032140 -10.471  < 2e-16 ***
bs(dis, df = 4, knots = c(3, 6, 9))5 -0.269575   0.058799  -4.585 5.75e-06 ***
bs(dis, df = 4, knots = c(3, 6, 9))6 -0.303386   0.062631  -4.844 1.70e-06 ***
---
Signif. codes:  0 �***� 0.001 �**� 0.01 �*� 0.05 �.� 0.1 � � 1

Residual standard error: 0.0612 on 499 degrees of freedom
Multiple R-squared:  0.7244,	Adjusted R-squared:  0.7211 
F-statistic: 218.6 on 6 and 499 DF,  p-value: < 2.2e-16
\end{verbatim}

\subsection*{e}
The error decreased as the degrees of freedom increased:\\
\begin{verbatim}
1.840173 1.833966 1.829884 1.816995 1.825653 1.792535 1.796992 1.788999
\end{verbatim}
 
\end{document}
