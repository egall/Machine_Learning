\documentclass{article}

\usepackage[latin1]{inputenc}
\usepackage{times,fullpage,amsmath}
\usepackage{enumerate,graphicx,hyperref,verbatim, amsmath, mathtools,pdfpages}
\DeclareMathSizes{10}{10}{10}{10}

\title{CS242 Homework \#3}

\author{esteggall, mcboivin, seanblum}

\date{Fall 2014}
\setlength\parindent{0pt}
\begin{document} \maketitle \pagestyle{empty}
\section*{Problem 1}
Degrees of freedom = $n - 1 = 9$\\

The t-statistic paired t-statistic formula is:\\\\
$\frac{\bar{X}_{D} - 0}{S_{D}/\sqrt{n}}$\\\\
Where  $\bar{X}_{D}$ is the mean of the difference of both samples, which is $0.05$, $S_{D}$ is the standard deviation of the difference between both samples, which is $0.085$ for us, and $n$ is the sample size, which in this case is 10.\\\\
Plugging in our values we get:\\\\
$\frac{0.5}{0.85/\sqrt{10}} = 1.86$\\\\
% verified using R
In order to determine the chance that A would appear this much better we need to find the p-value.\\
We calculated our p-value using R and got:\\\\
$p-value = 0.04786$\\\\
To determine if this is acceptable we consult the significance leves. (Found on wikipedia)\\
\begin{itemize}
 \item $p \leq 0.01$  : very strong presumption against null hypothesis\\
 \item $0.01 < p \leq 0.05$  : strong presumption against null hypothesis\\
 \item $0.05 < p \leq 0.1$ : low presumption against null hypothesis\\
 \item $p > 0.1$ : no presumption against the null hypothesis\\
\end{itemize}
According to this, we have a strong presumption aginst the null hypothesis, so our chances our good that the differences between the two samples are sampling error, however it is close to the threshold of 0.05, so we can't be too confident.\\

\section*{Problem 2}
\subsection*{a}
% Did you guys get funky ass alien characters when you loaded the .arff? wtf
% Think this is what he means by model:
Model is:\\
\verb|LinearRegression -S 0 -R 1.0E-8|\\
The root mean squared error we got is 0.1897.\\
\subsection*{b}
We can observer that Weka gives us the following formula:\\\\
$y = 4.36 + (-0.13*x_{1}) + (1.84*x_{2}) + (-0.89*x_{3})$\\\\
Plugging in our x's we get:\\\\
$y = 4.36 + (-0.13*3) + (1.84*3) + (-0.89*5)$\\\\
$y = 4.36 + (-0.13*3) + (1.84*3) + (-0.89*5)$\\\\
$y = 4.36 - 0.39 + 5.52 - 4.45 = 5.04$\\\\
\subsection*{c}
We got the values of:\\
\begin{verbatim}
           [,1]
[1,]  4.3608029
[2,] -0.1342938
[3,]  1.8476838
[4,] -0.8965848
\end{verbatim}
Which are identical to the coefficients from part A.\\
\subsection*{d}
There are no changes to the values of the coefficients.\\

\subsection*{e}
The values of the coefficients are pretty close to the coefficients from part A, except the weight belonging to $x_{1}$ is non existent. This is because the ridge parameter evidently cancels out the value of the coefficient that would belong to $x_{1}$.\\

\section*{Problem 3}
% http://perso.uclouvain.be/francois.glineur/files/papers/OptEng10a-final.pdf
\subsection*{a}
\subsection*{b}
\subsection*{c}
\section*{Problem 4}
% http://cbcb.umd.edu/~hcorrada/PML/homeworks/HW04_solutions.pdf
\subsection*{a}
\begin{tabular}{| l | c | r |}
 \hline      
 $w*x + b$ & $\bar{y}$ & L \\ \hline
  10 & +1 & 0.0000655 \\ \hline      
  4 & +1 & 0.026 \\ \hline      
  10 & +1 & 14.42 \\ \hline  
   4 & +1 & 5.8 \\ \hline            
\end{tabular}
\subsection*{b}
$ \frac{\partial }{\partial b} \frac{1}{\ln 2}\ln(1 + e^{-y(w*x + b)}) = $\\\\
$ \frac{1}{\ln 2} \frac{-y*e^{-y(w*x +b)}}{1 + e^{-y(w*x+b)}}$\\\\
\subsection*{c}
$ \frac{\partial }{\partial w} \frac{1}{\ln 2}\ln(1 + e^{-y(w*x + b)}) = $\\\\
$ \frac{1}{\ln 2} \frac{-yx*e^{-y(w*x +b)}}{1 + e^{-y(w*x+b)}}$\\\\
\subsection*{d}
\begin{tabular}{| c | c | c | c | c | c | c | c | c | c | c | c |}
\hline
$w_{1}$ & $w_{2}$ & b & $x_{1}$ & $x_{2}$ & y & $w*x + b$ & $\bar{y}$ & L & $\frac{\partial{l}}{\partial{w_{1}}}$ & $\frac{\partial{l}}{\partial{w_{2}}}$ & $\frac{\partial{l}}{\partial{b}}$\\ \hline
1 & 2 & 3 & 3 & -1 & -1 & 4 & +1 & 5.8 & 4.25 & -2.83 & 1.417 \\ \hline
1.425 & 1.717 & 3.142 & 3 & -1 & -1 & 5.699 & +1 & 8.227 & - & - & - \\ \hline
\end{tabular}
\end{document}

\end{document}